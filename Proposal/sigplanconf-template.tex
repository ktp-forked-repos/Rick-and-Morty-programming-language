%-----------------------------------------------------------------------------
%
%               Template for sigplanconf LaTeX Class
%
% Name:         sigplanconf-template.tex
%
% Purpose:      A template for sigplanconf.cls, which is a LaTeX 2e class
%               file for SIGPLAN conference proceedings.
%
% Guide:        Refer to "Author's Guide to the ACM SIGPLAN Class,"
%               sigplanconf-guide.pdf
%
% Author:       Paul C. Anagnostopoulos
%               Windfall Software
%               978 371-2316
%               paul@windfall.com
%
% Created:      15 February 2005
%
%-----------------------------------------------------------------------------


\documentclass[numbers]{sigplanconf}

% The following \documentclass options may be useful:

% preprint      Remove this option only once the paper is in final form.
% 10pt          To set in 10-point type instead of 9-point.
% 11pt          To set in 11-point type instead of 9-point.
% numbers       To obtain numeric citation style instead of author/year.

\usepackage{amsmath}
\usepackage{enumitem}
\PassOptionsToPackage{hyphens}{url}\usepackage{hyperref}

\newcommand{\cL}{{\cal L}}

\begin{document}

\special{papersize=8.5in,11in}
\setlength{\pdfpageheight}{\paperheight}
\setlength{\pdfpagewidth}{\paperwidth}

\conferenceinfo{CONF 'yy}{Month d--d, 20yy, City, ST, Country}
\copyrightyear{20yy}
\copyrightdata{978-1-nnnn-nnnn-n/yy/mm}
\copyrightdoi{nnnnnnn.nnnnnnn}

% Uncomment the publication rights you want to use.
%\publicationrights{transferred}
%\publicationrights{licensed}     % this is the default
%\publicationrights{author-pays}

\titlebanner{}        % These are ignored unless
\preprintfooter{}   % 'preprint' option specified.

\title{Linear Temporal programming in an Esoteric Programming Language}
\subtitle{Contact: [email]@grinnell.edu}


\authorinfo{Tristan Knoth}
           {Grinnell College}
           {[knothtri17]}
\authorinfo{Reilly Grant}
           {Grinnell College}
           {[grantrei]}
\authorinfo{Chris Kottke}
           {Grinnell College}
           {[kottkech17]}

%remove ACM copyright stuff
\makeatletter
\def\@copyrightspace{\relax}
\makeatother
%

\maketitle

\begin{abstract}
For this project we will develop an esoteric programming language
based on the popular cartoon Rick and Morty. This programing
language will support multithreading and will also make use of integrated Linear Temporal logic to verify the
correctness of multithreaded programs. In addition, this language
is intended to be accessible enough to attract students and
programmers unfamiliar with the concepts of temporal logic and parallelism.
\end{abstract}

%Unsure what the purpose of this is
%\category{CR-number}{subcategory}{third-level}

%\keywords
%keyword1, keyword2

\section{Introduction}
The popular TV show Rick and Morty uses the concepts of
parallel timelines and alternate universes as a main theme in many
episodes. Given the popularity of the show and its relevance in pop
culture, we decided that it would be an appropriate medium through
which to introduce the concepts of multithreading and temporal logic.
The language's syntax will include many references to the show,
and thus be attractive to fans of the show. It will also make the use of
temporal logic more whimsical and thus increase the topic's appeal to a diverse audience.

Our final goal is to create a Turing complete language which satisfies
the above conditions of reference, and also makes serious use of linear temporal programming and its related concepts.

\section{Prior Work}
 

Linear Temoral Logic is a mathematical framework that deals with
statements whose truth value can change over time. It has been used in programming for
software verification, specificly in when working with concurrent
programs\cite{AutomaticVerification, SimpleVerification}. Temporal
Logic has also been used in declarative programming languages to
increase the expressiveness\cite{DeclarativeTemporal}. We could not find a non declarative programming language that
supports temporal logic. 

Additional work that will be important to this project comes from the field of language design.
Language design is a field which has had a lot of focus directed
towards it. There are currently hundreds of languages that
have been created purely to explore the bounds of programming languages, and the field of programming language research has been growing rapidly in recent years \cite{esolang}. It is easy to find advice on
creating programming languages, such as that given by Dominic Orchard
of the university of Cambridge \cite{4Rs}.  With the large amount of work
that has been done on programming languages, we predict that finding
resources to assist us will not be extraordinarily difficult.

\section{Proposed Work}
Temporal logic can be an extremely useful concept in various 
computing problems, particularly related to program verification. 
In order to ensure, for example, that a certain state is eventually
reached,
 we need temporal logic. Similarly, we can use temporal logic to
 ensure
% that certain eventualities are never reached. A final important
 application is to ensure "fairness" in multi-threaded systems. A program may, 
for example, want to ensure that if a process makes a certain request 
often enough, that request is eventually fulfilled.

Thus, we propose to use Haskell to develop a simple Turing-complete
 computer language featuring temporal logic and some simple
 multi-threading
 utilities. Without multi-threading, we cannot use the temporal logic 
capabilities to their full potential. GHC already supports implicit
parallelism.
 Thus, we can simply use GHC's implementation as the foundation
 of our multi-threading capabilities.
 The primary goal of our use of multithreading is not user control;
 rather it will give a context in which temporal logic can be more
 usefully applied. While we are unsure of exactly how we will
 implement this system, threadpools are a particularly attractive
 option as they will allow easy "blocking" and management of 
 threaded resources.

Our language will also support various temporal logic statements. 
This will allow programmers to properly verify non-terminating applications. 
\section{Timeline}
\begin{enumerate}
\item    Friday 11/4: project checkpoint 1 due:

  By Friday 11/4, we intend to have a basic Turing complete language
  with many of the jokes and references of Rick and Morty Implemented.

\item  Friday 11/18: project checkpoint 2 due:

By Friday 11/18 we intend to have implemented singly threaded
temporally logical system integrated into the language. This will be 
reminiscent of Haskell's lazy execution due to it's singly threaded nature.

\item  Friday 12/2: project checkpoint 3 due:
  
  By Friday 12/2 we intend to have easily implemented a system which
  allows the user to simply handle multithreading, as well as serial execution.

\item  Monday 12/5 and Wednesday 12/7: project presentations:

By this point, we intend to have finished the project, and mostly be
cleaning it presenting to our peers.

\item Friday 12/9: final project deliverables due:

By this point, we will have the basic all implemented, or at least
some versions of all of them. We may have remaining features that we
were unable to implement due to the scope of the project, but complete
basics implementation will be done.
\end{enumerate}

\nocite{*}

%\acks
%Acknowledgments, if needed.

% We recommend abbrvnat bibliography style.

\bibliographystyle{abbrvnat}
\bibliography{sigplanconf}{}







\end{document}
